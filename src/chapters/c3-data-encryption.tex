\chapter{Data Encryption in Linux Systems}
\label{chapter:intro}

\section{eCryptfs}
\label{sec:de-ecryptfs}

As mentioned before, eCryptfs is a stacked file system which manages data encryption. Now, details about it and the advantages and disadvantages is brings will be presented.

Originally authored by Michael Halcrow and the IBM LInux Technology Center, eCryptfs is derived from Erez Zadok's Cryptfs, and the FiST framework for stacked filesystems. eCryptfs extended Cryptfs to provide advanced key management and policy features. 

\subsection{Block Encryption Vs. Stacked File System Encryption}
\label{sub-sec:be-vs-sfse}

The two approaches to disk encryption are block-based and file-based. Block encryption means that the actual encryption process happens when the filesystem writes a block of data on the disk. Its advantages are simplicity and transparency. However, the method lacks granularity, i.e., treating each file differently. This is the type of encryption used in Android 3.0 and later.

A more in detail comparison of block-based encryption and file-based encryption is presented in the following table.\footnote{\url{http://ksouedu.com/doc/ecryptfs-utils/ecryptfs-faq.html\#compare}}

\renewcommand{\arraystretch}{1.8}
\begin{table}[tp]
	\begin{tabularx}{\textwidth}{|| m{0.46\textwidth} || m{0.46\textwidth} ||}
		\hhline{|t:=:t:=:t|}
		\multicolumn{1}{||c||}{\textbf{Block Device Encryption}} & 
			\multicolumn{1}{c||}{\textbf{Stacked Filesystem Encryption}} \\ 
		\hhline{|:=::=:|}
		Simple in concept and implementation; just transform blocks as they pass through. & High level of design complexity; meticulous handling of internal filesystem primitives required. \\
		\hhline{|:=::=:|}
		Must allocate a block device to dedicate for the entire filesystem. & Stacks on top of existing mounted filesystems; requires no special on-disk storage allocation effort. \\
		\hhline{|:=::=:|}
		Everything in the filesystem incurs the cost of encryption and decryption, regardless of the confidentiality requirements for the data. & Selective encryption of the contents of only the sensitive files. \\
		\hhline{|:=::=:|}
		Fully protects the confidentiality of the directory structures, superblocks, file sizes, file permissions, and so forth. & Cannot keep all filesystem metadata confidential. Since stacked filesystems encrypt on a per-file basis, attackers will know the approximate file sizes, for instance. \\
		\hhline{|:=::=:|}
		Coarse granularity; only fixed per-mountpoint encryption policies are possible. & Fine granularity; flexible per-file encryption policies are possible. \\
		\hhline{|:=::=:|}		
		No notion of ``encrypted files.'' Individual files must be re-encrypted via a userspace application before written to backups, sent via email, etc. & Individual encrypted files can be accessed transparently by applications; no additional work needed on the part of applications before moving the files to another location. \\
		\hhline{|:=::=:|}
		Clients cannot use directly on networked filesystems; encryption must be set up and managed on the server, or the client must encase all of his files in a loopback mount, losing the per-file granularity from the perspective of other clients. & Clients can stack on locally mounted networked filesystems; individual files are sent to the server and stored in encrypted form. \\
		\hhline{|:=::=:|}
		Can protect databases that use their own dedicated block device. & Can only protect databases that write their tables to regular files in an existing filesystem. \\
		\hhline{|:=::=:|}
		Used to protect swap space. & Not designed to protect swap space; we recommend using block device encryption to protect swap space while using eCryptfs on the filesystem. \\
		\hhline{|:=::=:|}
		Possible to hide the fact that the partition is encrypted. & The fact that encrypted data exists on the device is obvious to an observer. \\
		\hhline{|:=::=:|}
		Filesystem-agnostic; any filesystem will work on an encrypted block device. & Can only be expected to work with existing filesystems that are upstream in the official Linux kernel. \\
		\hhline{|b:=:b:=:b|}
	\end{tabularx}
	\caption{Block Device Encryption vs Stacked Filesystem Encryption}
\end{table}
